\TraceSemaine{Semaine 2}{9 Février 2026}{
    
    \buts{Mettre en place un environnement de développement (Workflow Git) et sutructurer le projet.}
    
    \fait{
        \item Initialisation du dépôt GitHub \texttt{TIPE-2026-Groupes} avec une bonne architecture modulaire (séparation Code / Journal / Biblio).
        \item Refonte du Journal LaTeX : découpage du \texttt{main.tex} en sous-fichiers pour faciliter la rédaction long-terme.
        \item Configuration de l'IDE Cloud (\textbf{GitHub Codespaces}) : installation des extensions LaTeX Workshop et des dépendances Linux (\texttt{texlive}, \texttt{chktex}).
        \item Premier déploiement réussi du PDF compilé via VS Code Web.
    }
    
    \problemes{
        Configuration de l'authentification Git (fin du support des mots de passe) $\to$ Génération d'un \textit{Personal Access Token}. 
        Débogage de l'environnement LaTeX (paquets manquants pour \texttt{tcolorbox} et erreurs de syntaxe sur les caractères spéciaux \texttt{\&}).
    }
    
    \suite{Commencer à faire les recherches sur le sujet des groupes kleinéens et rédiger les premières définitions dans le journal.}
}